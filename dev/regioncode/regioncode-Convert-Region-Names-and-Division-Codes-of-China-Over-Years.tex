% Options for packages loaded elsewhere
\PassOptionsToPackage{unicode}{hyperref}
\PassOptionsToPackage{hyphens}{url}
\PassOptionsToPackage{dvipsnames,svgnames,x11names}{xcolor}
%
\documentclass[
  article]{jss}

\usepackage{amsmath,amssymb}
\usepackage{iftex}
\ifPDFTeX
  \usepackage[T1]{fontenc}
  \usepackage[utf8]{inputenc}
  \usepackage{textcomp} % provide euro and other symbols
\else % if luatex or xetex
  \usepackage{unicode-math}
  \defaultfontfeatures{Scale=MatchLowercase}
  \defaultfontfeatures[\rmfamily]{Ligatures=TeX,Scale=1}
\fi
\usepackage{lmodern}
\ifPDFTeX\else  
    % xetex/luatex font selection
\fi
% Use upquote if available, for straight quotes in verbatim environments
\IfFileExists{upquote.sty}{\usepackage{upquote}}{}
\IfFileExists{microtype.sty}{% use microtype if available
  \usepackage[]{microtype}
  \UseMicrotypeSet[protrusion]{basicmath} % disable protrusion for tt fonts
}{}
\makeatletter
\@ifundefined{KOMAClassName}{% if non-KOMA class
  \IfFileExists{parskip.sty}{%
    \usepackage{parskip}
  }{% else
    \setlength{\parindent}{0pt}
    \setlength{\parskip}{6pt plus 2pt minus 1pt}}
}{% if KOMA class
  \KOMAoptions{parskip=half}}
\makeatother
\usepackage{xcolor}
\setlength{\emergencystretch}{3em} % prevent overfull lines
\setcounter{secnumdepth}{-\maxdimen} % remove section numbering
% Make \paragraph and \subparagraph free-standing
\ifx\paragraph\undefined\else
  \let\oldparagraph\paragraph
  \renewcommand{\paragraph}[1]{\oldparagraph{#1}\mbox{}}
\fi
\ifx\subparagraph\undefined\else
  \let\oldsubparagraph\subparagraph
  \renewcommand{\subparagraph}[1]{\oldsubparagraph{#1}\mbox{}}
\fi


\providecommand{\tightlist}{%
  \setlength{\itemsep}{0pt}\setlength{\parskip}{0pt}}\usepackage{longtable,booktabs,array}
\usepackage{calc} % for calculating minipage widths
% Correct order of tables after \paragraph or \subparagraph
\usepackage{etoolbox}
\makeatletter
\patchcmd\longtable{\par}{\if@noskipsec\mbox{}\fi\par}{}{}
\makeatother
% Allow footnotes in longtable head/foot
\IfFileExists{footnotehyper.sty}{\usepackage{footnotehyper}}{\usepackage{footnote}}
\makesavenoteenv{longtable}
\usepackage{graphicx}
\makeatletter
\def\maxwidth{\ifdim\Gin@nat@width>\linewidth\linewidth\else\Gin@nat@width\fi}
\def\maxheight{\ifdim\Gin@nat@height>\textheight\textheight\else\Gin@nat@height\fi}
\makeatother
% Scale images if necessary, so that they will not overflow the page
% margins by default, and it is still possible to overwrite the defaults
% using explicit options in \includegraphics[width, height, ...]{}
\setkeys{Gin}{width=\maxwidth,height=\maxheight,keepaspectratio}
% Set default figure placement to htbp
\makeatletter
\def\fps@figure{htbp}
\makeatother

\usepackage{orcidlink,thumbpdf,lmodern}

\newcommand{\class}[1]{`\code{#1}'}
\newcommand{\fct}[1]{\code{#1()}}
\usepackage{xeCJK}
\makeatletter
\makeatother
\makeatletter
\makeatother
\makeatletter
\@ifpackageloaded{caption}{}{\usepackage{caption}}
\AtBeginDocument{%
\ifdefined\contentsname
  \renewcommand*\contentsname{Table of contents}
\else
  \newcommand\contentsname{Table of contents}
\fi
\ifdefined\listfigurename
  \renewcommand*\listfigurename{List of Figures}
\else
  \newcommand\listfigurename{List of Figures}
\fi
\ifdefined\listtablename
  \renewcommand*\listtablename{List of Tables}
\else
  \newcommand\listtablename{List of Tables}
\fi
\ifdefined\figurename
  \renewcommand*\figurename{Figure}
\else
  \newcommand\figurename{Figure}
\fi
\ifdefined\tablename
  \renewcommand*\tablename{Table}
\else
  \newcommand\tablename{Table}
\fi
}
\@ifpackageloaded{float}{}{\usepackage{float}}
\floatstyle{ruled}
\@ifundefined{c@chapter}{\newfloat{codelisting}{h}{lop}}{\newfloat{codelisting}{h}{lop}[chapter]}
\floatname{codelisting}{Listing}
\newcommand*\listoflistings{\listof{codelisting}{List of Listings}}
\makeatother
\makeatletter
\@ifpackageloaded{caption}{}{\usepackage{caption}}
\@ifpackageloaded{subcaption}{}{\usepackage{subcaption}}
\makeatother
\makeatletter
\makeatother
\ifLuaTeX
  \usepackage{selnolig}  % disable illegal ligatures
\fi
\IfFileExists{bookmark.sty}{\usepackage{bookmark}}{\usepackage{hyperref}}
\IfFileExists{xurl.sty}{\usepackage{xurl}}{} % add URL line breaks if available
\urlstyle{same} % disable monospaced font for URLs
\hypersetup{
  pdftitle={regioncode: Convert Region Names and Division Codes of China Over Years},
  pdfauthor={Yue Hu; Wenquan Wu},
  pdfkeywords={regioncode, geocodes, administrative division
codes, linguistic zone, Pinyin},
  colorlinks=true,
  linkcolor={blue},
  filecolor={Maroon},
  citecolor={Blue},
  urlcolor={Blue},
  pdfcreator={LaTeX via pandoc}}

%% -- Article metainformation (author, title, ...) -----------------------------

%% Author information
\author{Yue Hu\\Tsinghua University \And Wenquan Wu\\Tsinghua
University,}
\Plainauthor{Yue Hu, Wenquan Wu} %% comma-separated

\title{regioncode: Convert Region Names and Division Codes of China Over
Years}
\Plaintitle{regioncode: Convert Region Names and Division Codes of China
Over Years} %% without formatting

%% an abstract and keywords
\Abstract{The Chinese government gives unique geocodes for each county,
city (prefecture), and provincial-level administrative unit.
\texttt{regioncode} provides a convenient way to convert Chinese
administrative division codes, official names, sociopolitical and
linguistic areas, abbreviations, and so on between each other.}

%% at least one keyword must be supplied
\Keywords{regioncode, geocodes, administrative division
codes, linguistic zone, Pinyin}
\Plainkeywords{regioncode, geocodes, administrative division
codes, linguistic zone, Pinyin}

%% publication information
%% NOTE: Typically, this can be left commented and will be filled out by the technical editor
%% \Volume{50}
%% \Issue{9}
%% \Month{June}
%% \Year{2012}
%% \Submitdate{2012-06-04}
%% \Acceptdate{2012-06-04}
%% \setcounter{page}{1}
%% \Pages{1--xx}

%% The address of (at least) one author should be given
%% in the following format:
\Address{
Yue Hu\\
E-mail: \email{yuehu@tsinghua.edu.cn}\\
\\~
Wenquan Wu\\
E-mail: \email{wuwq20@mails.tsinghua.edu.cn}\\
\\~

}

\begin{document}
\maketitle
\hypertarget{why-regioncode}{%
\section{Why regioncode?}\label{why-regioncode}}

The Chinese government gives unique geocodes for each county, city
(prefecture), and provincial-level administrative unit. These
``administrative division codes'' are consistently
\href{http://www.mca.gov.cn/article/sj/xzqh/1980/}{adjusted and updated}
to matched national and regional plans of development. The adjustments
however may disturb researchers when they conduct studies over time or
merge geo-based data from different years. Especially, when researchers
render statistical data on a Chinese map, different geocodes between map
data and statistical data can cause mess-up outputs.

This package aims to conquer such difficulties by a one-step solution.
In the current version, \texttt{regioncode} enables seamlessly
converting formal names, common-used names, language zone, and division
codes of Chinese provinces and prefectures between each other and across
thirty-four years from 1986 to 2019.

\hypertarget{installation}{%
\section{Installation}\label{installation}}

To install:

\begin{itemize}
\tightlist
\item
  the latest released version: \texttt{install.packages("regioncode")}.
\item
  the latest developing version:
  \texttt{remotes::install\_github("sammo3182/regioncode")}.
\end{itemize}

\hypertarget{basic-usage}{%
\section{Basic Usage}\label{basic-usage}}

We uses a randomly drawn sample of Yuhua Wang's
\href{https://dataverse.harvard.edu/dataset.xhtml?persistentId=doi:10.7910/DVN/9QZRAD}{\texttt{China’s\ Corruption\ Investigations\ Dataset}}
to illustrate how the package works. In the data, the division codes
were recorded with the 2019 version, and we added prefectural
abbreviations for the sake of illustration.

In \texttt{regioncode} package, we named administrative division codes
as \texttt{code}, regions' formal names as \texttt{name}, and their
commonly used abbreviation as \texttt{sname}. The current version
enables mutual conversion between any pair of them. To do so, users just
need to pass a character vector of names or a numeric vector of geocodes
into the function. In the current version, the function can produce
three types of output at both the prefectural and provincial levels:
codes (\texttt{code}), names (\texttt{name}) and area (\texttt{area},
such as 华北, 东北, 华南, etc.). One just needs to specify which type of
the output they want in the argument \texttt{convert\_to} and
corresponding years of the input and output. For example, the following
codes convert the 2019 geocodes in the \texttt{corruption} data to their
1989 version:

\begin{verbatim}
library(regioncode)
\end{verbatim}

\begin{verbatim}
Warning: package 'regioncode' was built under R version 4.3.1
\end{verbatim}

\begin{verbatim}
data("corruption")

# Original 2019 version
corruption$prefecture_id
\end{verbatim}

\begin{verbatim}
 [1] 370100 321200 310117 420500 451300 431200 350300 511500 469021 420600
\end{verbatim}

\begin{verbatim}
# 1999 version
regioncode(data_input = corruption$prefecture_id, 
           convert_to = "code", # default set
           year_from = 2019,
           year_to = 1989)
\end{verbatim}

\begin{verbatim}
Joining with `by = join_by(`2019_code`)`
\end{verbatim}

\begin{verbatim}
 [1] 370100 329001 310227 420500 422700 452200 433000 350300 512500 460025
[11] 420600
\end{verbatim}

Note that if a region was initially geocoded in e.g., 1989 and included
in a new region, in 2019, the new region geocode will be used hereafter.
If a big place was broken into several regions, the later-year codes
will be aligned with the first region according to the ascendant order
of the regions' numeric geocodes.

By altering the output format to \texttt{name}, one can easily convert
codes or region names of a given year to region names in another year.
\texttt{regioncode} automatically detects the input format, so users
need to specify the \emph{output} format only (together with the input
and output years) to gain what they want. In the following example, we
convert the geocode variable in the \texttt{corruption} dataset to
region names and the name variable to codes and names in another year.

\begin{verbatim}
# The original name
corruption$prefecture
\end{verbatim}

\begin{verbatim}
 [1] "济南市" "泰州市" "松江区" "宜昌市" "来宾市" "怀化市" "莆田市" "宜宾市"
 [9] "定安县" "襄阳市"
\end{verbatim}

\begin{verbatim}
# Codes to name

regioncode(data_input = corruption$prefecture_id, 
           convert_to = "name",
           year_from = 2019,
           year_to = 1989)
\end{verbatim}

\begin{verbatim}
Joining with `by = join_by(`2019_code`)`
\end{verbatim}

\begin{verbatim}
 [1] "济南市"   "泰州市"   "松江县"   "宜昌市"   "宜昌地区" "柳州地区"
 [7] "怀化地区" "莆田市"   "宜宾地区" "定安县"   "襄樊市"  
\end{verbatim}

\begin{verbatim}
# Name to codes of the same year
regioncode(data_input = corruption$prefecture, 
           convert_to = "code",
           year_from = 2019,
           year_to = 2019)
\end{verbatim}

\begin{verbatim}
Joining with `by = join_by(`2019_name`)`
\end{verbatim}

\begin{verbatim}
 [1] 370100 321200 310117 420500 451300 431200 350300 511500 469021 420600
\end{verbatim}

\begin{verbatim}
# Name to name of a different year

regioncode(data_input = corruption$prefecture, 
           convert_to = "name",
           year_from = 2019,
           year_to = 1989)
\end{verbatim}

\begin{verbatim}
Joining with `by = join_by(`2019_name`)`
\end{verbatim}

\begin{verbatim}
 [1] "济南市"   "泰州市"   "松江县"   "宜昌市"   "宜昌地区" "柳州地区"
 [7] "怀化地区" "莆田市"   "宜宾地区" "定安县"   "襄樊市"  
\end{verbatim}

\hypertarget{advanced-applications}{%
\section{Advanced Applications}\label{advanced-applications}}

To further help uses with ``messier'' data and diverse demands,
\texttt{regioncode} provides five special conversions: conversion from
data with incomplete data, specification of municipalities, conversion
sociopolitical areas and linguistic areas, and pinyin output. The
current version also allows conversions at the provincial level.

\hypertarget{incomplete-naming-prefectures.}{%
\subsection{Incomplete naming
prefectures.}\label{incomplete-naming-prefectures.}}

More than often, data codes may omit the administrative level when
recording geo-information, e.g., using ``北京'' instead of ``北京市.''
To accomplish conversions of such data, one needs to specify the
\texttt{incomplete\_name} argument. If the input data is incomplete,
users should set the argument as ``from''; if they want the output name
(when \texttt{convert\_to\ =\ "name"}) to be without ``city'' or
``prefecture,'' they can set the argument to ``to'' (see the example
below); and if users want to gain incomplete names for both input and
output names, \texttt{incomplete\_name\ =\ "both"}. All the above
conversions can be over years.

\begin{verbatim}
# Full, official names
corruption$prefecture
\end{verbatim}

\begin{verbatim}
 [1] "济南市" "泰州市" "松江区" "宜昌市" "来宾市" "怀化市" "莆田市" "宜宾市"
 [9] "定安县" "襄阳市"
\end{verbatim}

\begin{verbatim}
regioncode(data_input = corruption$prefecture, 
           convert_to = "name",
           year_from = 2019,
           year_to = 1989,
           incomplete_name = "to")
\end{verbatim}

\begin{verbatim}
Joining with `by = join_by(`2019_name`)`
\end{verbatim}

\begin{verbatim}
 [1] "济南" "泰州" "松江" "宜昌" "柳州" "来宾" "怀化" "莆田" "宜宾" "定安"
[11] "襄樊" "襄阳"
\end{verbatim}

\hypertarget{municipalities}{%
\subsection{Municipalities}\label{municipalities}}

Municipalities (``直辖市'') are geographically cities but
administratively provincial. The districts within these municipalities
are thus prefectural. Different analyses treat these districts
differently: some parallel the districts aligned with other prefectures,
while the others treat the entire municipality as one prefecture. To
deal with the latter situation, \texttt{regioncode} sets an argument
\texttt{zhixiashi}. When the argument is set \texttt{TRUE}, the
municipalities are treated as whole prefectures, and their provincial
codes are used as the geocodes.

\hypertarget{sociopolitical-and-linguistic-areas}{%
\subsection{Sociopolitical and Linguistic
Areas}\label{sociopolitical-and-linguistic-areas}}

Due to social, political, and martial reasons, Chinese regions are
divided into seven regions:

\begin{longtable}[]{@{}
  >{\raggedright\arraybackslash}p{(\columnwidth - 2\tabcolsep) * \real{0.1111}}
  >{\raggedright\arraybackslash}p{(\columnwidth - 2\tabcolsep) * \real{0.8889}}@{}}
\toprule\noalign{}
\begin{minipage}[b]{\linewidth}\raggedright
region
\end{minipage} & \begin{minipage}[b]{\linewidth}\raggedright
provincial-level administrative unit
\end{minipage} \\
\midrule\noalign{}
\endhead
\bottomrule\noalign{}
\endlastfoot
华北 & 北京市, 天津市, 山西省, 河北省, 内蒙古自治区 \\
东北 & 黑龙江省, 吉林省, 辽宁省 \\
华东 & 上海市, 江苏省, 浙江省, 安徽省, 福建省, 台湾省, 江西省, 山东省 \\
华中 & 河南省, 湖北省, 湖南省 \\
华南 & 广东省, 海南省, 广西壮族自治区, 香港特别行政区, 澳门特别行政区 \\
西南 & 重庆市, 四川省, 贵州省, 云南省, 西藏自治区 \\
西北 & 陕西省, 甘肃省, 青海省, 宁夏回族自治区, 新疆维吾尔自治区 \\
\end{longtable}

\texttt{regioncode} also offers a method ``area'' to convert codes and
names of the region into such areas.

\begin{verbatim}
regioncode(data_input = corruption$prefecture, 
           year_from = 2019,
           year_to = 1989, 
           convert_to="area")
\end{verbatim}

\begin{verbatim}
Joining with `by = join_by(`2019_name`)`
\end{verbatim}

\begin{verbatim}
 [1] "华东" "华东" "华东" "华中" "华南" "华中" "华东" "西南" "华南" "华中"
\end{verbatim}

China is a multilingual country with a variety of dialects. These
dialects may be used by several prefectures in a province or province.
Prefectures from different provinces may also share the same dialect.

\texttt{regioncode} allows users to gain linguistic zones the
prefectures belong as an output. Users can gain two levels of linguistic
zones, dialect groups and dialect sub-groups by setting the argument
\texttt{to\_pinyin} to \texttt{dia\_group} or \texttt{dia\_sub\_group}.
Note that, the linguistic distribution in China is too complex for
precisely gauging at the prefectural level. The linguistic zone output
from \texttt{regioncode} is thus at most for reference rather than
rigorous linguistic research.

\begin{verbatim}
regioncode(data_input = corruption$prefecture, 
           year_from = 2019,
           year_to = 1989,
           to_dialect = "dia_group")
\end{verbatim}

\begin{verbatim}
Joining with `by = join_by(`2019_name`)`
\end{verbatim}

\begin{verbatim}
 [1] "冀鲁官话" "江淮官话" "吴语"     "西南官话" "西南官话" "湘语"    
 [7] "莆仙区"   "西南官话" "琼文区"   "西南官话"
\end{verbatim}

\begin{verbatim}
regioncode(data_input = corruption$prefecture, 
           year_from = 2019,
           year_to = 1989,
           to_dialect = "dia_sub_group")
\end{verbatim}

\begin{verbatim}
Joining with `by = join_by(`2019_name`)`
\end{verbatim}

\begin{verbatim}
 [1] "沧惠片-1"  "石济片-8"  "泰如片-1"  "太湖片-1"  "成渝片-3"  "成渝片-9" 
 [7] "桂柳片-10" "岑江片-2"  "吉溆片-3"  "娄邵片-1"  "黔北片-3"  "长益片-3" 
[13] "莆仙区-4"  "灌赤片-10" "府城片-1"  "鄂北片-10"
\end{verbatim}

\hypertarget{pinyin}{%
\subsection{Pinyin}\label{pinyin}}

Pinyin is a Chinese phonetic romanization. Some data stores the region
names with pinyin instead of Chinese characters. The default name output
of \texttt{regioncode} uses Chinese characters, but one can gain pinyin
output by setting the argument \texttt{to\_pinyin\ =\ TRUE}. The effect
can be applied to either official name, incomplete name, or
sociopolitical area outputs.

\begin{verbatim}
regioncode(data_input = corruption$prefecture, 
           year_from = 2019,
           year_to = 1989, 
           convert_to="name",
           to_pinyin=TRUE
           )
\end{verbatim}

\begin{verbatim}
Joining with `by = join_by(`2019_name`)`
\end{verbatim}

\begin{verbatim}
           济南市            泰州市            松江县            宜昌市 
     "ji_nan_shi"    "tai_zhou_shi" "song_jiang_xian"    "yi_chang_shi" 
         宜昌地区          柳州地区          怀化地区            莆田市 
 "yi_chang_di_qu"  "liu_zhou_di_qu"  "huai_hua_di_qu"     "pu_tian_shi" 
         宜宾地区            定安县            襄樊市 
   "yi_bin_di_qu"    "ding_an_xian"   "xiang_fan_shi" 
\end{verbatim}

\begin{verbatim}
regioncode(data_input = corruption$prefecture, 
           year_from = 2019,
           year_to = 1989, 
           convert_to="name",
           incomplete_name = "to",
           to_pinyin=TRUE
           )
\end{verbatim}

\begin{verbatim}
Joining with `by = join_by(`2019_name`)`
\end{verbatim}

\begin{verbatim}
        济南         泰州         松江         宜昌         柳州         来宾 
    "ji_nan"   "tai_zhou" "song_jiang"   "yi_chang"   "liu_zhou"    "lai_bin" 
        怀化         莆田         宜宾         定安         襄樊         襄阳 
  "huai_hua"    "pu_tian"     "yi_bin"    "ding_an"  "xiang_fan" "xiang_yang" 
\end{verbatim}

\begin{verbatim}
regioncode(data_input = corruption$prefecture, 
           year_from = 2019,
           year_to = 1989, 
           convert_to="area",
           to_pinyin=TRUE
           )
\end{verbatim}

\begin{verbatim}
Joining with `by = join_by(`2019_name`)`
\end{verbatim}

\begin{verbatim}
       华东        华东        华东        华中        华南        华中 
 "hua_dong"  "hua_dong"  "hua_dong" "hua_zhong"   "hua_nan" "hua_zhong" 
       华东        西南        华南        华中 
 "hua_dong"    "xi_nan"   "hua_nan" "hua_zhong" 
\end{verbatim}

\hypertarget{provinces}{%
\subsection{Provinces}\label{provinces}}

\texttt{regioncode} allows conversions at not only the prefectural but
provincial level. By setting the argument \texttt{province\ =\ TRUE},
users can accomplish all the code, name, and area conversions at the
provincial level. (Note that, at the provincial level, the linguistic
conversion can be only to dialect group.) Moreover, since provinces have
fixed abbreviations, \texttt{regioncode} allows names not only being,
e.g., ``宁夏'' instead of ``宁夏回族自治区'' but also ``宁''. When the
inputs are abbreviations, users can set the \texttt{convert\_to}
argument to \texttt{abbreTocode}, \texttt{abbreToname}, or
\texttt{abbreToarea}. When they want provincial abbreviation outputs,
just set \texttt{convert\_to\ =\ "abbre"}.

\begin{verbatim}
regioncode(data_input = corruption$province_id, 
           convert_to = "codeToabbre",
           year_from = 2019,
           year_to = 1989,
           province = TRUE)
\end{verbatim}

\begin{verbatim}
Joining with `by = join_by(prov_code)`
\end{verbatim}

\begin{verbatim}
 [1] "鲁" "苏" "沪" "鄂" "桂" "湘" "闽" "蜀" "琼" "鄂"
\end{verbatim}

\hypertarget{conclusion}{%
\section{Conclusion}\label{conclusion}}

\texttt{regioncode} provides a convenient way to convert Chinese
administrative division codes, official names, sociopolitical and
linguistic areas, abbreviations, and so on between each other. This
vignette offers a quick view of package features and a short tutorial
for users.

The development of the package is ongoing. Future versions aim to add
more administrative level choices, from province level to county level.
Data are also enriching. Welcome to join us if you are also interested
(see the affiliations below). Please contact us with any questions or
comments. Bug reports can be conducted by
\href{https://github.com/sammo3182/regioncode/issues}{Github Issues}.

Also thanks ZHU Meng and LIU Xueyan for helping writing the `Advanced
Application' section of this vignette.

\hypertarget{sec-references}{%
\section*{References}\label{sec-references}}
\addcontentsline{toc}{section}{References}



\end{document}
